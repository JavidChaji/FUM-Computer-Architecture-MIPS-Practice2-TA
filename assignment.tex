
\def \Subject {تمرین سری دوم}
\def \Course {درس معماری کامپیوتر}
\def \Report {تمرین MIPS سری دوم}
% \def \StudentNumber {98524044}

\begin{center}
\vspace{.4cm}
{\bf {\huge \Subject}}\\
{\bf \Large \Course}
\vspace{.2cm}
\end{center}
% {\bf \Author }  \\
% \hspace{} 
{\bf مهلت تحویل تا: \date{1402/02/23}}    
\hspace{\fill} 
{\bf \Large \Report} \\
\hrule
\vspace{0.8cm}

%\huge{\Subject}\\[1.5 cm]
%\chapterauthor{\Author~ : \StudentNumber}
\begin{itemize}
    \item این تمرین به صورت گروهی است. (گروه های دو نفره) 
    \item برای انجام تمرین دیدن ویدئوهای آموزش کار با syscall ها و آموزش کار با حافظه پویا که در ویو وجود دارد توصیه می‌شود.
    \item دقت داشته باشید که برای چاپ مقادیر در خروجی و همین طور برای گرفتن ورودی از کاربر باید با syscall ها آشنا باشید.
    \item برای تمرین مرتب سازی ادغامی نیز کار با Stack نیاز است بنابراین دیدن ویدئوی آموزش کار با حافظه‌ی پویا مفید خواهد بود.
    \item  پاسخ تمرین را به صورت یک فایل با فرمت : \\\lr{\small{FirstnameLastname\_StudentNumber\_FirstnameLastname\_StudentNumber.zip}}\\ بارگذاری کنید.
    \\(مثال (MohammadMohammadi\_9XXXXXX\_RezaRezaei\_9XXXXXX.zip
    \item کامنت گذاری تمام خطوط کد الزامی است.
    \item تحویل تکلیف بعد از مهلت مشخص شده نمره ای نخواهد داشت.
    \item در صورت اثبات کپی برداری، نمره تکالیف کپی شده و کپی شونده هر دو از ۱۰۰ نمره، ۱۰۰- خواهد بود.
    \item زمانبندی تحویل آنلاین تمرین پس از اتمام مهلت ارسال اعلام خواهد شد.
    \item تحویل تمرینات از طریق تلگرام، ایمیل و ... امکان پذیر نیست.
\end{itemize}

\section{محیط دایره}
{به زبان MIPS Assembly کدی بنویسید که محیط دایره با اندازه شعاع دلخواه اندازه گیری کند.} \\
{(\begin{math}\pi \approx 3\end{math}, دقت کنید که مقدار شعاع انتخابی عددی حسابی باشد.)}\\
\bf {نکات :}
\normalfont
\begin{enumerate}
    \item {ورودی که همان شعاع دایره است به دلخواه شما می تواند در قسمت داده ها \lr{(.data segment)} تعریف و یا از کاربر با چاپ پیام مناسب از طریق کنسول گرفته شود.}
    \item {خروجی برنامه با چاپ پیغامی مناسب در کنسول باید بگوید که محیط دایره با شعاع ورودی چه مقداری دارد.}
\end{enumerate}
\colorbox{green}{پاسخ:}\\
\normalfont

{در این کد، مقادیر شعاع و \begin{math}\pi \end{math} را به ترتیب در رجیسترهای \$s0 و \$s1 از حافظه بارگذاری می‌کنیم. سپس محیط دایره را با ضرب 2، \begin{mah}\pi \end{math} و شعاع با استفاده از دستور mul محاسبه می‌کنیم. نتیجه را در رجیستر \$t0 ذخیره می‌کنیم. در نهایت، با استفاده از دستور syscall با پارامترهای مناسب، محیط دایره را چاپ می‌کنیم. سپس با استفاده از دستور دیگری به نام syscall با کد خروجی مناسب، برنامه را خاتمه می‌دهیم.}\\
{توجه: این کد فرض می‌کند که مقدار \begin{math}\pi \end{math} برابر با 3 است. در صورت نیاز به دقت بیشتر، مقدار \begin{math}\pi \end{math} در حافظه باید به‌روز شود.}
\begin{latin}
\begin{listing}[ht]
    \inputminted[linenos=true]{asm}{sources/Circumference_of_the_Circle.mips}
    \caption{Circumference of the Circle}
    \label{listing:}
\end{listing}
\end{latin}


% 2
\clearpage
\section{تبدیل اعداد دودویی به دهدهی}
{به زبان MIPS Assembly کدی بنویسید که اعداد دودویی (binary) را به اعداد دهدهی (decimal) تبدیل کند.}\\
\bf {نکات :}
\normalfont
\begin{enumerate}
    \item {ورودی همان عدد مورد نظر در مبنای دو هست که باید در قسمت داده ها \lr{(.data segment)} تعریف شود. (می توان تعداد ارقام عدد ورودی را هم در ورودی تعیین کرد.)}
    \item {خروجی برنامه با چاپ پیغامی مناسب در کنسول باید معادل ده‌دهی عدد باینری ورودی را چاپ کند.}
\end{enumerate}
\colorbox{green}{پاسخ:}\\
\normalfont

{این کد MIPS، یک عدد دودویی که به صورت یک رشته در حافظه ذخیره شده، به معادل ده‌دهی آن تبدیل می کند.
ابتدا آدرس رشته دودویی در حافظه در \$s0 قرار داده می شود و سپس طول رشته در \$t0 ذخیره می شود.
سپس یک حلقه شروع می شود که به تمامی کاراکترهای رشته دودویی می رود و هر کدام را به معادل ده‌دهی آن تبدیل می کند.
مقدار ده‌دهی به صورت پیوسته به وسیله یک حلقه در \$t1 ذخیره می شود.
در نهایت، مقدار ده‌دهی به وسیله سیستم کال بر روی کنسول چاپ می شود.}

\begin{latin}
\begin{listing}[ht]
    \inputminted[linenos=true]{asm}{sources/Binary_Number_to_Decimal.mips}
    \caption{Binary Number to Decimal}
    \label{listing:}
\end{listing}
\end{latin}


% 3
\clearpage    
\section{پیداکردن بزرگترین مقسوم علیه مشترک}
{به زبان MIPS Assembly کدی بنویسید که بزرگترین مقسوم علیه مشترک دو عدد را محاسبه کند.}\\
\bf {نکات :}
\normalfont
\begin{enumerate}
    \item {ورودی که دو عدد مورد نظر برای محاسبه بزرگترین مقسوم علیه مشترک آن ها است به دلخواه شما می تواند در قسمت داده ها \lr{(.data segment)} تعریف و یا از کاربر با چاپ پیام مناسب از طریق کنسول گرفته شود.}
    \item {خروجی برنامه با چاپ پیغامی مناسب در کنسول بزرگترین مقسوم علیه مشترک دو عدد ورودی را چاپ می کند.}
    \item {فرض می کنیم اعداد ورودی فقط اعداد حسابی خواهد بود.}
\end{enumerate}
\colorbox{green}{پاسخ:}\\
\normalfont

{این برنامه دو عدد را دریافت کرده و بزرگترین مقسوم علیه آن‌ها را پیدا می‌کند و چاپ می‌کند. برای این کار ابتدا دو عدد با استفاده از دستور lw از حافظه خوانده می‌شود و سپس یک حلقه برای محاسبه مقسوم علیه بزرگترین شروع می‌شود. در هر مرحله در ابتدا بررسی می‌شود که آیا یکی از این دو عدد صفر است؟ اگر یکی از آن‌ها صفر باشد، عدد دیگر مقسوم علیه بزرگترین است. در غیر این صورت، دو عدد بررسی می‌شوند تا بزرگترین را پیدا کنند و سپس از کوچک‌ترین به بزرگ‌ترین عدد کم می‌شود. اگر عدد اول کوچک‌تر باشد، آن‌ها جا به جا می‌شوند تا در مرحله بعدی عدد اول بزرگتر باشد. پس از پیدا کردن مقسوم علیه بزرگترین، پیغامی رشته‌ای که نشان‌دهنده مقسوم علیه بزرگترین است، با استفاده از دستور li و la چاپ می‌شود. سپس مقسوم علیه بزرگترین با استفاده از دستور move در \$t1 قرار داده می‌شود و در نهایت یک کاراکتر خط جدید چاپ می‌شود و برنامه خاتمه می‌یابد.}

\begin{latin}
\begin{listing}[ht]
    \inputminted[linenos=true]{asm}{sources/GCD_of_Two_Numbers.mips}
    \caption{GCD of Two Numbers}
    \label{listing:}
\end{listing}
\end{latin}


% 4
\clearpage
\section{محاسبه فاکتوریل}
{به زبان MIPS Assembly کدی بنویسید که فاکتوریل یک عدد را محاسبه کند .}\\
\bf {نکات :}
\normalfont
\begin{enumerate}
    \item {ورودی که عدد مورد نظر برای محاسبه فاکتوریل آن است به دلخواه شما می تواند در قسمت داده ها \lr{(.data segment)} تعریف و یا از کاربر با چاپ پیام مناسب از طریق کنسول گرفته شود.}
    \item {خروجی برنامه با چاپ پیغامی مناسب در کنسول فاکتوریل عدد ورودی را چاپ می کند.}
    \item {فرض می کنیم اعداد ورودی فقط اعداد طبیعی خواهد بود.}
\end{enumerate}
\colorbox{green}{پاسخ:}\\
\normalfont

{این کد برنامه‌ای را برای محاسبه فاکتوریل یک عدد غیر منفی نوشته است.
اولین بخش کد، بخش داده‌ها است. در این بخش، سه رشته حاوی متن‌های \lr{"Enter a non-negative integer: "}، \lr{"The factorial is: "} و \lr{"\textbackslash n"} تعریف شده است.
بخش دیگر کد، بخش متن است. در این بخش، تابع main نوشته شده است. این تابع با چاپ متن \lr{"Enter a non-negative integer: "} کار خود را شروع می‌کند و عدد ورودی کاربر را دریافت می‌کند. سپس، متغیرهای \$t1 و \$t2 به ترتیب برای نگه‌داری فاکتوریل و شمارنده حلقه اولیه مقداردهی می‌شوند.
سپس، در حلقه‌ای با استفاده از دستورات بررسی شرطی (bgt)، ضرب (mul) و جمع یکی (addi)، فاکتوریل عدد ورودی محاسبه می‌شود. در هر دور از حلقه، شمارنده حلقه به مقدار یکی اضافه می‌شود و بررسی می‌شود که آیا شمارنده از عدد ورودی بزرگتر شده است یا خیر؟ در صورتی که شمارنده از عدد ورودی بزرگتر شده باشد، از حلقه خارج شده و فاکتوریل محاسبه شده را چاپ می‌کند.
در پایان، برنامه با دستور \lr{li \$v0, 10} و syscall خاتمه می‌یابد و اجرای برنامه به پایان می‌رسد.}

\begin{latin}
\begin{listing}[ht]
    \inputminted[linenos=true]{asm}{sources/Factorial_of_a_Number.mips}
    \caption{Factorial of a Number}
    \label{listing:}
\end{listing}
\end{latin}
 

% 5
\clearpage
\section{مرتب‌سازی ادغامی}
{به زبان MIPS Assembly کدی بنویسید که با گرفتن اندازه و اعضای آرایه مورد نظر، آرایه را به روش مرتب سازی ادغامی مرتب کند.}\\
\bf {نکات :}
\normalfont
\begin{enumerate}
    \item {ورودی که تعداد اعضای آرایه و خود آرایه مورد نظر برای مرتب کردن است باید از کاربر با چاپ پیام مناسب از طریق کنسول گرفته شود.}
    \item {خروجی برنامه با چاپ پیغامی مناسب در کنسول آرایه مرتب شده را چاپ می کند.}
    \item {فرض می کنیم اعداد ورودی فقط اعداد حسابی خواهد بود.}
\end{enumerate}
\colorbox{green}{پاسخ:}\\
\normalfont

{این برنامه ابتدا از کاربر تعداد عناصر آرایه را دریافت می‌کند، سپس عناصر آرایه را از کاربر می‌خواهد و در یک آرایه اولیه ذخیره می‌کند. در ادامه، برنامه با استفاده از تابع Mergesort آرایه را مرتب می‌کند و سپس آرایه مرتب شده را چاپ می‌کند.
در بخش .data ابتدا سه آرایه و چند متغیر تعریف شده است، این آرایه‌ها برای نگهداری موقتی داده‌ها هستند و متغیرها برای ذخیره اطلاعاتی مانند تعداد عناصر آرایه و اندازه آرایه و آدرس پایان آرایه تعریف شده‌اند.
در بخش .text ابتدا برنامه از کاربر تعداد عناصر آرایه را دریافت می‌کند و سپس عناصر آرایه را در یک حلقه از کاربر می‌خواهد و در آرایه‌ای اولیه ذخیره می‌کند.
سپس آدرس پایان آرایه و اندازه آرایه محاسبه شده و تابع Mergesort صدا زده می‌شود.
تابع Mergesort به طور بازگشتی آرایه را به دو بخش تقسیم می‌کند تا زمانی که بخش‌های آرایه یک عنصر داشته باشند، سپس این بخش‌ها را به صورت مرتب شده با هم ترکیب می‌کند.
تابع Merge نیز دو زیرآرایه را به یکدیگر می‌چسباند و نتیجه را در آرایه‌ی اصلی ذخیره می‌کند.
در نهایت، تابع printArray آرایه را چاپ می‌کند.}

\begin{latin}
\begin{listing}[ht]
    \inputminted[linenos=true, firstline=1, lastline=55]{asm}{sources/Merg_Sort.mips}
    \caption{Merg Sort}
    \label{listing:}
\end{listing}

\begin{listing}[ht]
  \inputminted[linenos=true, firstline=56, lastline=110]{asm}{sources/Merg_Sort.mips}  
    \caption{Merg Sort}
    \label{listing:}
\end{listing}

\begin{listing}[ht]
  \inputminted[linenos=true, firstline=111, lastline=165]{asm}{sources/Merg_Sort.mips}  
    \caption{Merg Sort}
    \label{listing:}
\end{listing}

\begin{listing}[ht]
  \inputminted[linenos=true, firstline=166, lastline=215]{asm}{sources/Merg_Sort.mips}  
    \caption{Merg Sort}
    \label{listing:}
\end{listing}

\begin{listing}[ht]
  \inputminted[linenos=true, firstline=216, lastline=last]{asm}{sources/Merg_Sort.mips}  
    \caption{Merg Sort}
    \label{listing:}
\end{listing}
\end{latin}
